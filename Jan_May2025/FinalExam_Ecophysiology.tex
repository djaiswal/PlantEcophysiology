% Options for packages loaded elsewhere
\PassOptionsToPackage{unicode}{hyperref}
\PassOptionsToPackage{hyphens}{url}
%
\documentclass[
]{article}
\usepackage{amsmath,amssymb}
\usepackage{iftex}
\ifPDFTeX
  \usepackage[T1]{fontenc}
  \usepackage[utf8]{inputenc}
  \usepackage{textcomp} % provide euro and other symbols
\else % if luatex or xetex
  \usepackage{unicode-math} % this also loads fontspec
  \defaultfontfeatures{Scale=MatchLowercase}
  \defaultfontfeatures[\rmfamily]{Ligatures=TeX,Scale=1}
\fi
\usepackage{lmodern}
\ifPDFTeX\else
  % xetex/luatex font selection
\fi
% Use upquote if available, for straight quotes in verbatim environments
\IfFileExists{upquote.sty}{\usepackage{upquote}}{}
\IfFileExists{microtype.sty}{% use microtype if available
  \usepackage[]{microtype}
  \UseMicrotypeSet[protrusion]{basicmath} % disable protrusion for tt fonts
}{}
\makeatletter
\@ifundefined{KOMAClassName}{% if non-KOMA class
  \IfFileExists{parskip.sty}{%
    \usepackage{parskip}
  }{% else
    \setlength{\parindent}{0pt}
    \setlength{\parskip}{6pt plus 2pt minus 1pt}}
}{% if KOMA class
  \KOMAoptions{parskip=half}}
\makeatother
\usepackage{xcolor}
\usepackage[margin=1in]{geometry}
\usepackage{color}
\usepackage{fancyvrb}
\newcommand{\VerbBar}{|}
\newcommand{\VERB}{\Verb[commandchars=\\\{\}]}
\DefineVerbatimEnvironment{Highlighting}{Verbatim}{commandchars=\\\{\}}
% Add ',fontsize=\small' for more characters per line
\usepackage{framed}
\definecolor{shadecolor}{RGB}{248,248,248}
\newenvironment{Shaded}{\begin{snugshade}}{\end{snugshade}}
\newcommand{\AlertTok}[1]{\textcolor[rgb]{0.94,0.16,0.16}{#1}}
\newcommand{\AnnotationTok}[1]{\textcolor[rgb]{0.56,0.35,0.01}{\textbf{\textit{#1}}}}
\newcommand{\AttributeTok}[1]{\textcolor[rgb]{0.13,0.29,0.53}{#1}}
\newcommand{\BaseNTok}[1]{\textcolor[rgb]{0.00,0.00,0.81}{#1}}
\newcommand{\BuiltInTok}[1]{#1}
\newcommand{\CharTok}[1]{\textcolor[rgb]{0.31,0.60,0.02}{#1}}
\newcommand{\CommentTok}[1]{\textcolor[rgb]{0.56,0.35,0.01}{\textit{#1}}}
\newcommand{\CommentVarTok}[1]{\textcolor[rgb]{0.56,0.35,0.01}{\textbf{\textit{#1}}}}
\newcommand{\ConstantTok}[1]{\textcolor[rgb]{0.56,0.35,0.01}{#1}}
\newcommand{\ControlFlowTok}[1]{\textcolor[rgb]{0.13,0.29,0.53}{\textbf{#1}}}
\newcommand{\DataTypeTok}[1]{\textcolor[rgb]{0.13,0.29,0.53}{#1}}
\newcommand{\DecValTok}[1]{\textcolor[rgb]{0.00,0.00,0.81}{#1}}
\newcommand{\DocumentationTok}[1]{\textcolor[rgb]{0.56,0.35,0.01}{\textbf{\textit{#1}}}}
\newcommand{\ErrorTok}[1]{\textcolor[rgb]{0.64,0.00,0.00}{\textbf{#1}}}
\newcommand{\ExtensionTok}[1]{#1}
\newcommand{\FloatTok}[1]{\textcolor[rgb]{0.00,0.00,0.81}{#1}}
\newcommand{\FunctionTok}[1]{\textcolor[rgb]{0.13,0.29,0.53}{\textbf{#1}}}
\newcommand{\ImportTok}[1]{#1}
\newcommand{\InformationTok}[1]{\textcolor[rgb]{0.56,0.35,0.01}{\textbf{\textit{#1}}}}
\newcommand{\KeywordTok}[1]{\textcolor[rgb]{0.13,0.29,0.53}{\textbf{#1}}}
\newcommand{\NormalTok}[1]{#1}
\newcommand{\OperatorTok}[1]{\textcolor[rgb]{0.81,0.36,0.00}{\textbf{#1}}}
\newcommand{\OtherTok}[1]{\textcolor[rgb]{0.56,0.35,0.01}{#1}}
\newcommand{\PreprocessorTok}[1]{\textcolor[rgb]{0.56,0.35,0.01}{\textit{#1}}}
\newcommand{\RegionMarkerTok}[1]{#1}
\newcommand{\SpecialCharTok}[1]{\textcolor[rgb]{0.81,0.36,0.00}{\textbf{#1}}}
\newcommand{\SpecialStringTok}[1]{\textcolor[rgb]{0.31,0.60,0.02}{#1}}
\newcommand{\StringTok}[1]{\textcolor[rgb]{0.31,0.60,0.02}{#1}}
\newcommand{\VariableTok}[1]{\textcolor[rgb]{0.00,0.00,0.00}{#1}}
\newcommand{\VerbatimStringTok}[1]{\textcolor[rgb]{0.31,0.60,0.02}{#1}}
\newcommand{\WarningTok}[1]{\textcolor[rgb]{0.56,0.35,0.01}{\textbf{\textit{#1}}}}
\usepackage{longtable,booktabs,array}
\usepackage{calc} % for calculating minipage widths
% Correct order of tables after \paragraph or \subparagraph
\usepackage{etoolbox}
\makeatletter
\patchcmd\longtable{\par}{\if@noskipsec\mbox{}\fi\par}{}{}
\makeatother
% Allow footnotes in longtable head/foot
\IfFileExists{footnotehyper.sty}{\usepackage{footnotehyper}}{\usepackage{footnote}}
\makesavenoteenv{longtable}
\usepackage{graphicx}
\makeatletter
\def\maxwidth{\ifdim\Gin@nat@width>\linewidth\linewidth\else\Gin@nat@width\fi}
\def\maxheight{\ifdim\Gin@nat@height>\textheight\textheight\else\Gin@nat@height\fi}
\makeatother
% Scale images if necessary, so that they will not overflow the page
% margins by default, and it is still possible to overwrite the defaults
% using explicit options in \includegraphics[width, height, ...]{}
\setkeys{Gin}{width=\maxwidth,height=\maxheight,keepaspectratio}
% Set default figure placement to htbp
\makeatletter
\def\fps@figure{htbp}
\makeatother
\setlength{\emergencystretch}{3em} % prevent overfull lines
\providecommand{\tightlist}{%
  \setlength{\itemsep}{0pt}\setlength{\parskip}{0pt}}
\setcounter{secnumdepth}{-\maxdimen} % remove section numbering
\ifLuaTeX
  \usepackage{selnolig}  % disable illegal ligatures
\fi
\usepackage{bookmark}
\IfFileExists{xurl.sty}{\usepackage{xurl}}{} % add URL line breaks if available
\urlstyle{same}
\hypersetup{
  pdftitle={Plant Ecophysiology FInal Exam},
  hidelinks,
  pdfcreator={LaTeX via pandoc}}

\title{Plant Ecophysiology FInal Exam}
\author{}
\date{\vspace{-2.5em}2025-04-29}

\begin{document}
\maketitle

\subsection{Soybean Growth Simulations Using Default Parameters and
Input (Soil/Climate)
Drivers}\label{soybean-growth-simulations-using-default-parameters-and-input-soilclimate-drivers}

Here I am loading R package BioCro and running simulations to grow
soybean using default weather and soil data in the package. Additionally
all the parameters and modules are already set up

\begin{Shaded}
\begin{Highlighting}[]
\FunctionTok{library}\NormalTok{(BioCro)}
\FunctionTok{library}\NormalTok{(lattice)}
\FunctionTok{library}\NormalTok{(knitr)}
\NormalTok{result }\OtherTok{\textless{}{-}} \FunctionTok{with}\NormalTok{(soybean, \{}\FunctionTok{run\_biocro}\NormalTok{(}
\NormalTok{  initial\_values,}
\NormalTok{  parameters,}
\NormalTok{  soybean\_weather}\SpecialCharTok{$}\StringTok{\textquotesingle{}2002\textquotesingle{}}\NormalTok{,}
\NormalTok{  direct\_modules,}
\NormalTok{  differential\_modules,}
\NormalTok{  ode\_solver}
\NormalTok{)\})}
\FunctionTok{xyplot}\NormalTok{(Stem }\SpecialCharTok{+}\NormalTok{ Leaf }\SpecialCharTok{+}\NormalTok{ Root }\SpecialCharTok{+}\NormalTok{ lai }\SpecialCharTok{+}\NormalTok{ Grain }\SpecialCharTok{\textasciitilde{}}\NormalTok{ TTc, }\AttributeTok{data =}\NormalTok{ result, }\AttributeTok{type=}\StringTok{\textquotesingle{}l\textquotesingle{}}\NormalTok{, }\AttributeTok{auto =} \ConstantTok{TRUE}\NormalTok{)}
\end{Highlighting}
\end{Shaded}

\includegraphics{FinalExam_Ecophysiology_files/figure-latex/cars-1.pdf}
You can see that X-axix is TTc (instead of days or time). This variable
(TTc) is thermal time and it is a different way to express time in
thermal units. Thermal time measures how much useful warmth a plant
experiences over time. It's calculated by summing the daily temperatures
that exceed a certain base temperature (the minimum temperature required
for growth). This accumulation is typically expressed in degree-days
\#\# Initial Conditions and Input Drivers Required for running Soybean
Simulations

You can see that several inputs are required to run the simulations. The
initial\_values list contains the starting values for all parameters
that need to be initialized---for example, the initial soil moisture
level. The parameters list includes the values of all model parameters
necessary for running the simulation, such as Vcmax, which represents
the maximum carboxylation rate of Rubisco at 25°C.

The data frame soybean\_weather\$`2002' provides the environmental input
data (weather drivers) for the simulation. The direct and differential
modules contain a set of sub-models (e.g., for simulating
photosynthesis, light interception, etc.) that together represent
whole-plant growth dynamics.

The ode\_solver component allows you to select a numerical method for
solving the system of ordinary differential equations (ODEs), which are
used to model the time-evolving behavior of plant growth.

\begin{Shaded}
\begin{Highlighting}[]
\CommentTok{\#initial value of soil moisture}
\NormalTok{initial\_soil\_moisture }\OtherTok{\textless{}{-}}\NormalTok{ soybean}\SpecialCharTok{$}\NormalTok{initial\_values}\SpecialCharTok{$}\NormalTok{soil\_water\_content}
\CommentTok{\#maximum carboxylation rate of rubisco}
\NormalTok{Vmax }\OtherTok{\textless{}{-}}\NormalTok{ soybean}\SpecialCharTok{$}\NormalTok{parameters}\SpecialCharTok{$}\NormalTok{vmax1}
\CommentTok{\#Weather driver}
\FunctionTok{head}\NormalTok{(soybean\_weather}\SpecialCharTok{$}\StringTok{\textquotesingle{}2002\textquotesingle{}}\NormalTok{)}
\end{Highlighting}
\end{Shaded}

\begin{verbatim}
##      year doy hour   zen solar netsolar dw_solar up_solar temp   rh windspeed
## 3625 2002 152    0 117.0     0     0.00      0.0     0.00 23.7 0.48     1.060
## 3626 2002 152    1 114.0     0     0.00      0.0     0.00 23.0 0.56     1.180
## 3627 2002 152    2 108.0     0     0.00      0.0     0.00 22.6 0.61     0.665
## 3628 2002 152    3 100.0     0     0.01      0.0     0.00 21.7 0.72     2.170
## 3629 2002 152    4  90.5    24     9.78     10.3     3.96 20.9 0.68     3.230
## 3630 2002 152    5  80.3    60    20.80     23.0     6.49 20.9 0.70     2.720
##      precip day_length time_zone_offset
## 3625      0       14.3               -6
## 3626      0       14.3               -6
## 3627      0       14.3               -6
## 3628      0       14.3               -6
## 3629      0       14.3               -6
## 3630      0       14.4               -6
\end{verbatim}

\begin{Shaded}
\begin{Highlighting}[]
\CommentTok{\#Module to describe partitioning coefficients}
\NormalTok{soybean}\SpecialCharTok{$}\NormalTok{direct\_modules}\SpecialCharTok{$}\NormalTok{partitioning\_coefficients}
\end{Highlighting}
\end{Shaded}

\begin{verbatim}
## [1] "BioCro:partitioning_coefficient_logistic"
\end{verbatim}

\subsection{Getting more information abour module governing partitioning
coefficients}\label{getting-more-information-abour-module-governing-partitioning-coefficients}

As you can see that the module used for simulating partitioning process
is partitioning\_coefficient\_logistic.

To get more details about the partitioning module use function
module\_info

\begin{Shaded}
\begin{Highlighting}[]
\FunctionTok{module\_info}\NormalTok{(}\StringTok{"BioCro:partitioning\_coefficient\_logistic"}\NormalTok{)}
\end{Highlighting}
\end{Shaded}

\begin{verbatim}
## 
## 
## Module name:
##   partitioning_coefficient_logistic
## 
## Module input quantities:
##   DVI
##   alphaRoot
##   alphaStem
##   alphaLeaf
##   alphaShell
##   betaRoot
##   betaStem
##   betaLeaf
##   betaShell
##   kRhizome_emr
## 
## Module output quantities:
##   kRoot
##   kStem
##   kLeaf
##   kShell
##   kGrain
##   kRhizome
## 
## Module type (differential or direct):
##   direct
## 
## Requires a fixed step size Euler ode_solver:
##   no
\end{verbatim}

The module partitioning\_coefficient\_logistic takes ten input
parameters (e.g., DVI, alphaRoot, alphaStem, etc.) and produces six
output variables (kRoot, kStem, etc.), calculated at each time step. At
any given time, the sum of all k values equals 1, ensuring mass balance
in carbon allocation.

For example, if kRoot = 1 at a particular time step, it means that 100\%
of the assimilated carbon (via photosynthesis) is allocated to root
growth, and no carbon is allocated to other components like stems or
leaves.

To understand how parameters like alphaRoot, alphaStem, etc., influence
the values of kRoot, kStem, and others, refer to the following paper: 🔗
\url{https://gmd.copernicus.org/articles/8/1139/2015/}

You can also try modifying the parameter values to experiment and
observe how the output allocation changes over time.

\begin{Shaded}
\begin{Highlighting}[]
\NormalTok{parameters\_with\_modified\_partitioning }\OtherTok{\textless{}{-}}\NormalTok{ soybean}\SpecialCharTok{$}\NormalTok{parameters}
\NormalTok{parameters\_with\_modified\_partitioning}\SpecialCharTok{$}\NormalTok{alphaLeaf }\OtherTok{=} \DecValTok{30} \CommentTok{\# original value was 23.8895}
\NormalTok{parameters\_with\_modified\_partitioning}\SpecialCharTok{$}\NormalTok{betaLeaf }\OtherTok{=} \SpecialCharTok{{-}}\DecValTok{20} \CommentTok{\# original value was {-}18.0851}
\NormalTok{parameters\_with\_modified\_partitioning}\SpecialCharTok{$}\NormalTok{alphaStem }\OtherTok{=} \DecValTok{30} \CommentTok{\# original value was 24.8248}
\NormalTok{parameters\_with\_modified\_partitioning}\SpecialCharTok{$}\NormalTok{betaStem }\OtherTok{=} \SpecialCharTok{{-}}\DecValTok{20} \CommentTok{\# original value was  {-}18.5214}
\NormalTok{parameters\_with\_modified\_partitioning}\SpecialCharTok{$}\NormalTok{alphaRoot }\OtherTok{=} \DecValTok{30} \CommentTok{\# original value was 36.967}
\NormalTok{parameters\_with\_modified\_partitioning}\SpecialCharTok{$}\NormalTok{betaRoot }\OtherTok{=} \SpecialCharTok{{-}}\DecValTok{20} \CommentTok{\# original value was  {-}40.1915}
\NormalTok{parameters\_with\_modified\_partitioning}\SpecialCharTok{$}\NormalTok{alphaShell }\OtherTok{=} \DecValTok{30} \CommentTok{\# original value was 13.5955}
\NormalTok{parameters\_with\_modified\_partitioning}\SpecialCharTok{$}\NormalTok{betaShell }\OtherTok{=} \SpecialCharTok{{-}}\DecValTok{20} \CommentTok{\# original value was  {-}9.9809}

\NormalTok{result\_modified }\OtherTok{\textless{}{-}}\NormalTok{ result }\OtherTok{\textless{}{-}} \FunctionTok{with}\NormalTok{(soybean, \{}\FunctionTok{run\_biocro}\NormalTok{(}
\NormalTok{  initial\_values,}
  \AttributeTok{parameters =}\NormalTok{ parameters\_with\_modified\_partitioning,}
\NormalTok{  soybean\_weather}\SpecialCharTok{$}\StringTok{\textquotesingle{}2002\textquotesingle{}}\NormalTok{,}
\NormalTok{  direct\_modules,}
\NormalTok{  differential\_modules,}
\NormalTok{  ode\_solver}
\NormalTok{)\})}
\FunctionTok{xyplot}\NormalTok{(Stem }\SpecialCharTok{+}\NormalTok{ Leaf }\SpecialCharTok{+}\NormalTok{ Root }\SpecialCharTok{+}\NormalTok{ lai }\SpecialCharTok{+}\NormalTok{ Grain }\SpecialCharTok{\textasciitilde{}}\NormalTok{ TTc, }\AttributeTok{data =}\NormalTok{ result\_modified, }\AttributeTok{type=}\StringTok{\textquotesingle{}l\textquotesingle{}}\NormalTok{, }\AttributeTok{auto =} \ConstantTok{TRUE}\NormalTok{)}
\end{Highlighting}
\end{Shaded}

\includegraphics{FinalExam_Ecophysiology_files/figure-latex/unnamed-chunk-3-1.pdf}
You can see that the results differ significantly. This variation is due
to changes in the parameters controlling the biomass partitioning
modules, while all other parameters (e.g., those related to
photosynthesis, etc.) remain unchanged.

\subsection{Problem Statement}\label{problem-statement}

Compare the performance of two different algorithms (from separate R
packages) in optimizing the parameters---alphaLeaf, betaLeaf, alphaStem,
betaStem, alphaRoot, betaRoot, alphaShell, and betaShell---to achieve
the best match between the model predictions and the observed values
provided below.

\begin{Shaded}
\begin{Highlighting}[]
\NormalTok{observations }\OtherTok{\textless{}{-}} \FunctionTok{read.csv}\NormalTok{(}\StringTok{"\textasciitilde{}/observed\_data.csv"}\NormalTok{)}
\FunctionTok{kable}\NormalTok{(observations)}
\end{Highlighting}
\end{Shaded}

\begin{longtable}[]{@{}rrrrrr@{}}
\toprule\noalign{}
doy & Stem & Leaf & Root & Grain & lai \\
\midrule\noalign{}
\endhead
\bottomrule\noalign{}
\endlastfoot
160 & 0.13 & 0.180 & 0.15 & 0.000 & 0.600 \\
180 & 0.95 & 0.950 & 1.22 & 0.000 & 3.000 \\
200 & 1.70 & 1.600 & 6.40 & 0.000 & 5.500 \\
230 & 2.50 & 2.100 & 6.50 & 0.008 & 7.200 \\
250 & 2.60 & 1.000 & 5.80 & 2.120 & 3.300 \\
280 & 1.80 & 0.005 & 5.30 & 4.000 & 0.005 \\
\end{longtable}

o proceed, you'll need to define an objective function that calculates
the error between observed data and model predictions generated by
run\_biocro. Then, use optimization techniques to minimize this error by
tuning the parameters. You are encouraged to experiment with different
optimization algorithms to evaluate and compare their effectiveness in
identifying the best-fit parameter values.

\subsubsection{Parvat: NMOF \& nloptr}\label{parvat-nmof-nloptr}

\subsubsection{Swadha: roptim \& dfoptim}\label{swadha-roptim-dfoptim}

\subsubsection{Mausumi: GenSA \& graDiEnt}\label{mausumi-gensa-gradient}

\subsection{Evaluation Guidelines}\label{evaluation-guidelines}

Evaluation will be based on a small report. The report should contain

\begin{enumerate}
\def\labelenumi{\arabic{enumi}.}
\tightlist
\item
  A very brief introduction of the problem (in your own words) {[}20
  Marks{]}
\item
  A brief description of the two algorithm that you are going to work
  with {[}30 marks +30 marks{]}
\item
  Readibility and reproducibility of the code/script developed for
  optimization {[}30 marks + 30 marks{]}
\item
  A figure showing evolution of Error with computational time/iterations
  for both the algorithms {[}30 Marks{]}
\item
  A figure showing final (optimized/best) values of parameters obtained
  by the two algorithms{[}30 marks{]}
\item
  Anything creative that you can put in the final report {[}Bonus: 20
  Marks in addition to 200 (100\%){]}
\end{enumerate}

\end{document}
